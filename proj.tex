\documentclass{beamer}

\usepackage[russian]{babel}
\usepackage{paratype}

\begin{document}

\section{Недвижимость}

	\subsection{План}
	
		\begin{frame}
			\frametitle{Вложение в недвижимость}
			
			Доступный бюджет "--- \textbf{240 тыс.\ евро} или около \textbf{18 млн.\ руб.}
			
			\vspace{\baselineskip}
			\textbf{План действий:}
			\begin{enumerate}
			\item Регистрация ИП для снижения налоговой ставки при сдаче квартиры в аренду.
			\item Выбор «целевой аудитории» для сдачи, исходя из последующей ликвидности квартиры.
			\item Выбор подходящего района Москвы.
			\item Выбор конкретного объекта недвижимости с учётом финансовых возможностей целевой аудитории.
			\item Покупка квартиры и сдача в аренду.
			\item Продажа недвижимости по истечении 5 лет.
			\end{enumerate}
		
		\end{frame}

	\subsection{Регистрация ИП}
	
		\begin{frame}
			\frametitle{Зачем?}
		
			ИП может выбирать налоговый режим, одним из которых является УСНО, при которой выплачивается 6\% от зарабатываемой суммы.
			Физическое лицо обязано выплачивать НДФЛ в размере 13\%.
		
		\end{frame}

		\begin{frame}
			\frametitle{Какие ещё налоги платить и не платить?}
			
			\begin{itemize}
			
			\item Налог на имущество в размере 0.1\% от кадастровой стоимости квартиры в год.
			
			\item Ежегодный страховой взнос (с учётом его изменения, составляет примерно 40 тыс.\ руб.\ в год + 1\% от дохода свыше 300 тыс.\ рублей).
			
			\end{itemize}
		
		
		\end{frame}
		
		\begin{frame}
			\frametitle{}
		
		
		\end{frame}
		
		\begin{frame}
			\frametitle{}
		
		
		\end{frame}
		
		\begin{frame}
			\frametitle{}
		
		
		\end{frame}
		
		\begin{frame}
			\frametitle{}
		
		
		\end{frame}

\end{document}
\documentclass{beamer}

\usepackage[russian]{babel}
\usepackage{paratype}
\usepackage{xcolor}
\usepackage{array}
\usepackage{tabularx}
\usepackage{booktabs}

\usetheme{Darmstadt}
\usecolortheme{seahorse}

\newcommand{\col}{\textcolor[rgb]{0.2,0.,0.55}}

\title{{Сасалити}}
\author{Мишка, Вовка, Юлька, Янка и Сергей\\}
\date{\footnotesize{Москва "--- 2018}}

\AtBeginSection[]{
  \begin{frame}
  \vfill
  \centering
  \begin{beamercolorbox}[sep=8pt,center,shadow=true,rounded=true]{title}
    \usebeamerfont{title}\insertsectionhead\par%
  \end{beamercolorbox}
  \vfill
  \end{frame}
}

\AtBeginSubsection[]{
  \begin{frame}
  \vfill
  \centering
  \begin{beamercolorbox}[sep=8pt,center,shadow=true,rounded=true]{title}
    \usebeamerfont{title}\insertsubsectionhead\par%
  \end{beamercolorbox}
  \vfill
  \end{frame}
}

\begin{document}

\section{Недвижимость}

	\subsection{План}
	
		\begin{frame}
			\frametitle{Вложение в недвижимость}
			
			Доступный бюджет "--- \textbf{240 тыс.\ евро} или около \textbf{18 млн.\ руб.}
			
			\vspace{\baselineskip}
			\textbf{План действий:}
			\begin{enumerate}
			\item Регистрация ИП для снижения налоговой ставки при сдаче квартиры в аренду.
			\item Выбор «целевой аудитории» для сдачи, исходя из последующей ликвидности квартиры.
			\item Выбор подходящего района Москвы.
			\item Выбор конкретного объекта недвижимости с учётом финансовых возможностей целевой аудитории.
			\item Покупка квартиры и сдача в аренду.
			\item Продажа недвижимости по истечении 5 лет.
			\end{enumerate}
		
		\end{frame}

	\subsection{Регистрация ИП}
	
		\begin{frame}
			\frametitle{Зачем?}
		
			ИП может выбирать налоговый режим, одним из которых является УСНО, при которой выплачивается 6\% от зарабатываемой суммы.
			Физическое лицо обязано выплачивать НДФЛ в размере 13\%.
		
		\end{frame}

		\begin{frame}
			\frametitle{Какие ещё налоги платить и не платить?}
			
			\textbf{Платить:}
			\begin{itemize}
			
			\item Налог на имущество в размере 0.1\% от кадастровой стоимости квартиры в год.
			
			\item Ежегодный страховой взнос (с учётом его изменения, составляет примерно 40 тыс.\ руб.\ в год + 1\% от дохода свыше 300 тыс.\ рублей).
			
			\end{itemize}
		
			\textbf{Не платить:}
			\begin{itemize}
			
			\item Налог с продажи квартиры не выплачивается, если владение квартирой не менее пяти лет.
			
			\item Страховой взнос можно вычесть из налога.
			
			\end{itemize}
					
		\end{frame}
		
	\subsection{Целевая аудитория}
	
		\begin{frame}
			\frametitle{Кто вообще снимает квартиры?}
			
				\begin{itemize}
					\item Семьи "--- ненадежны и притязательны в цене.
					\item Рабочие, приезжие из стран СНГ "--- рискованно.
					\item Обеспеченные люди "--- слишком дорого для нас.
					\item \textbf{Студенты} "--- наш выбор.
							Снимают не по одиночке, съезжают только на два месяца летом, надежные.
				\end{itemize}
			
		\end{frame}
		
	
	\subsection{Выбор района покупки квартиры}
	
		
		\begin{frame}
			\frametitle{Стоимость квартиры (ноябрь 2018)}
		
			
				\centering
				\begin{tabular}{ l r }
					\toprule
					Округ                 & м\textsuperscript{2}\ (тыс.\ руб) \\ \midrule
					Центральный           & 297                               \\
					\textbf{Юго-Западный} & 200                               \\
					Западный              & 184                               \\
					Северо-Западный       & 174                               \\
					Северный              & 167                               \\
					Восточный             & 161                               \\
					Северо-Восточный      & 152                               \\
					Южный                 & 147                               \\
					Юго-Восточный         & 139                               \\ \bottomrule
				\end{tabular}
			

		
		\end{frame}
		
		\begin{frame}
			\frametitle{Как выбрать квартиру?}
		
			\begin{itemize}
			\item Небольшая удалённость от метро (не более 15 мин.\ пешком).
			\item Нормальный ремонт.
			\item Не 1 и последний этажи.
			\item Возможность размещения нескольких «независимых групп» студентов.
			\end{itemize}
			
			
		\end{frame}
		
		\begin{frame}
			\frametitle{Анализ Юго-Западного округа}
		
			\textbf{Какие прогнозы?}
			\begin{itemize}
			\item Изменение стоимости квартиры за год составляет от $-$4\% до 1.9\%.
			\item Уступает общемосковскому темп изменения цен на 0,5\% в месяц.
			\end{itemize}
			
			\textbf{Сколько стоит квартира?} Приемлемое жилье площадью около 60 кв.\ м.\ стоит 12--15 млн руб.
			
			\textbf{Как быстро сдаётся?} В течении недели.
			
			\textbf{Ликвидность.} Высокая.
			
		\end{frame}
		
		\begin{frame}
			\frametitle{Анализ Юго-Западного округа}
			\textbf{Конкретный пример:}
				\begin{itemize}
				\item 2-комн.\ квартира, 56,3 м\textsuperscript{2}.
				\item Москва, ЮЗАО, р-н Ломоносовский, просп.\ Вернадского,~15.
				\item 10 мин.\ пешком до метро Университет.
				\item этаж 3/9, недавний ремонт, дом 1958 года постройки, потолки 3,1~м.
				\item цена 13 млн, примерно равна кадастровой стоимости.
				\end{itemize}
			
		
		\end{frame}
		
		\begin{frame}
			\frametitle{}
		
		
		\end{frame}

	\section[Металлы]{Вложение в драгоценные металлы}

		\begin{frame}
			\frametitle{Вовкины слиточки}	
		
				180000 Евро = 13,273,200 рублей (5 декабря 2018 года Сбербанк)
							  
							  13641120 
							  
							  13,374,000 (30 ноября 2018 года Сбербанк)
				
				на 30 ноября 2018 года Сбербанк
				
				Стоимость слиточка золота 1000 гр == 2,755,000	рублей (ОМС)
				
				Стоимость слиточка палладия 100 гр == 275,000 рублей	(ОМС)	  
				
				Стоимость слиточка золота 1000 гр == 3,253,614	рублей 
				
				Стоимость слиточка палладия 100 гр == 327,391 рублей
				
				За последние три года стоимость золота выросла на $14\%$
				
				\qquad\qquad\qquad		пять лет на $102\%$	 в рубля и $0.64\%$ в долларах	
					
				За последние три года стоимость палладия выросла на $114\%$
				
				\qquad\qquad\qquad			пять лет на $234\%$	 в рубля и $61\%$ в долларах

		\end{frame}
		
		\begin{frame}
			\frametitle{Таблица за слиточек золото 1000гр, палладий 100гр}
			
				\centering
				\begin{tabular}{ l r r }
					\toprule
					Металл   & Рост (\%) & Приб./уб. (руб) \\ \midrule
					Золото   & 14        & 45980           \\
					         & 13        & 21410           \\
					         & 12        & -3160           \\
					         & 10        & -52300          \\[2mm]
					Палладий & 100       & 209000          \\
					         & 50        & 88000           \\
					         & 20        & 15400           \\
					         & 10        & -8800           \\ \bottomrule
				\end{tabular}
			
		\end{frame}

	\section{Вклады в банки}
		
		\begin{frame}
			\frametitle{Что важно?}
			
			\col{\textbf{Риски:}}
			\begin{itemize}
				\item Рост инфляции в стране
				\item Банк лишили лицензии
				\item «Тетрадочные» вклады
			\end{itemize}
			
			\col{\textbf{На что обратить внимание?}}
			\begin{itemize}
			\item Пополнение и снятие
			\item Выплата процентов
			\item Капитализация
			\end{itemize}
		\end{frame}

		
		\begin{frame}
			\frametitle{}
		
		
		\end{frame}
		
		\begin{frame}
			\frametitle{}
		
		
		\end{frame}
		
		\begin{frame}
			\frametitle{}
		
		
		\end{frame}
		
		\begin{frame}
			\frametitle{}
		
		
		\end{frame}

\end{document}